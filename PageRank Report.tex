\documentclass[a4paper, titlepage]{report}

\usepackage[latin1]{inputenc}
\usepackage[T1]{fontenc}
\usepackage{geometry} %marges
\usepackage[francais]{babel}
\usepackage{graphicx}
\usepackage{verbatim}
\usepackage{listings} %affichage du code
\usepackage[compact]{titlesec} % affichage compact
\usepackage{amsmath} % matrice

\title{SINF 1250: Rapport de projet}
\author{Maxime Dillion \and Jacques Yakoub}
\date{24 d�cembre 2017}

\pagestyle{headings}

\begin{document}

\maketitle
\tableofcontents

\chapter{Th�orie}
\section{Rappel}
\paragraph{}
[Rappel de la partie th�orique]

\subsection{Syst�me d'�quations lin�aires}

Avec cette m�thode, la recherche du vecteur de score devient une question de vecteur propre. Ici, il serait fait le choix d'expliquer le vecteur de droite afin de faciliter les calculs dont voici son expression math�matique :

% TODO pas familier avec les expressions math de latex ^^
Gt * X = X

Ou X �tant le vecteur et Gt la transpos�e de la matrice google.

Appliquons cette formule � la situation pr�sente :

\[
\begin{bmatrix}
\frac{1}{50} & \frac{187}{350} & \frac{49}{200} & \frac{1}{50} & \frac{47}{100} \\ \\
\frac{11}{50} & \frac{1}{50} & \frac{49}{200} & \frac{1}{5} & \frac{47}{100} \\ \\
\frac{21}{50} & \frac{1}{50} & \frac{1}{50} & \frac{37}{50} & \frac{1}{50} \\ \\
\frac{3}{25} & \frac{97}{350} & \frac{79}{200} & \frac{1}{50} & \frac{1}{50} \\ \\
\frac{11}{50} & \frac{26}{175} & \frac{19}{200} & \frac{1}{50} & \frac{1}{50}
\end{bmatrix}
*
\begin{bmatrix}
x1      \\
x2    	\\
x3		\\
x4		\\
x5		\\    
\end{bmatrix}
= 
\begin{bmatrix}
x1      \\
x2    	\\
x3		\\
x4		\\
x5		\\
\end{bmatrix} 
\]

En rajoutant une ligne de 1, nous nous assurons que la somme de x1,..,xn termes doit valoir 1. Enfin, avant de pouvoir r�soudre le syst�me par des op�rations matricielles pour obtenir une forme en escalier , il faut simplifier la forme pr�sente en passant les inconnues du c�t� gauche de l'�quation. 

% TODO A finir
\[
\begin{bmatrix}
\frac{-49}{50} & \frac{187}{350} & \frac{49}{200} & \frac{1}{50} & \frac{47}{100} \\ \\
\frac{11}{50} & \frac{-49}{50} & \frac{49}{200} & \frac{1}{5} & \frac{47}{100} \\ \\
\frac{21}{50} & \frac{1}{50} & \frac{-49}{50} & \frac{37}{50} & \frac{1}{50} \\ \\
\frac{3}{25} & \frac{97}{350} & \frac{79}{200} & \frac{-49}{50} & \frac{1}{50} \\ \\
\frac{11}{50} & \frac{26}{175} & \frac{19}{200} & \frac{1}{50} & \frac{-49}{50}
\end{bmatrix}
\]
	
\section{Calcul}
\subsection{Syst�me d'�quation sous forme matriciel}
% & \\
\[
\begin{pmatrix} 
1 & 1 & 1 & 1 & 1 & 1 \\ \\
\frac{-49}{50} & \frac{187}{350} & \frac{49}{200} & \frac{1}{50} & \frac{47}{100} & 0 \\ \\
\frac{11}{50} & \frac{-49}{50} & \frac{49}{200} & \frac{1}{5} & \frac{47}{100} & 0 \\ \\
\frac{21}{50} & \frac{1}{50} & \frac{-49}{50} & \frac{37}{50} & \frac{1}{50} & 0 \\ \\
\frac{3}{25} & \frac{97}{350} & \frac{79}{200} & \frac{-49}{50} & \frac{1}{50} & 0 \\ \\
\frac{11}{50} & \frac{26}{175} & \frac{19}{200} & \frac{1}{50} & \frac{-49}{50} & 0
\end{pmatrix}
\]

\subsection{Syst�me d'�quation apr�s r�solution}
\[
\begin{pmatrix} 
1 & 0 & 0 & 0 & 0 & \frac{278338891485003}{1185309949623550} \\ \\
0 & 1 & 0 & 0 & 0 & \frac{1574961688097759}{7540409867060228} \\ \\
0 & 0 & 1 & 0 & 0 & \frac{3332146709283619}{1.32265796170871e+16} \\ \\
0 & 0 & 0 & 1 & 0 & \frac{534623}{2789366} \\ \\
0 & 0 & 0 & 0 & 1 & \frac{1572003}{13946830} \\ \\
0 & 0 & 0 & 0 & 0 & 0
\end{pmatrix}
\]
On a donc :
$x_{1} = \frac{278338891485003}{1185309949623550}$ ;
$x_{2} = \frac{1574961688097759}{7540409867060228}$ ;
$x_{3} = \frac{3332146709283619}{1.32265796170871e+16}$ ;
$x_{4} = \frac{534623}{2789366}$ ;
$x_{5} = \frac{1572003}{13946830}$

\newpage

\chapter{Impl�mentation}

\section{Matrice d'adjacence}
\[
\begin{pmatrix} 
	0 & 2 & 4 & 1 & 2 \\
	4 & 0 & 0 & 2 & 1  \\
	3 & 3 & 0 & 5 & 1  \\
	0 & 1 & 4 & 0 & 0  \\
	3 & 3 & 0 & 0 & 0
\end{pmatrix}
\]

\section{Degr� entrant des noeuds}
\[
\begin{pmatrix} 
	10 & 9 & 8 & 8 & 4
\end{pmatrix}
\]

% TODO je pense qu'il faut mettre la matrice de probablit� P (et non Pt)
\section{Matrice de probabilit� de transition}
\[
\begin{pmatrix} 
0 & \frac{4}{7} & \frac{1}{4} & 0 & \frac{1}{2} \\
\frac{2}{9} & 0 & \frac{1}{4} & \frac{1}{5} & \frac{1}{2}  \\
\frac{4}{9} & 0 & 0 & \frac{4}{5} & 0  \\
\frac{1}{9} & \frac{2}{7} & \frac{5}{12} & 0 & 0  \\
\frac{2}{9} & \frac{1}{7} & \frac{1}{12} & 0 & 0
\end{pmatrix}
\]

\section{Matrice Google}
\[
\begin{pmatrix} 
\frac{1}{50} & \frac{11}{50} & \frac{21}{50} & \frac{3}{25} & \frac{11}{50} \\
\frac{187}{350} & \frac{1}{50} & \frac{1}{50} & \frac{97}{350} & \frac{26}{175}  \\
\frac{49}{200} & \frac{49}{200} & \frac{1}{50} & \frac{79}{200} & \frac{19}{200}  \\
\frac{1}{50} & \frac{1}{5} & \frac{37}{50} & \frac{1}{50} & \frac{1}{50}  \\
\frac{47}{100} & \frac{47}{100} & \frac{1}{50} & \frac{1}{50} & \frac{1}{50}
\end{pmatrix}
\]

\section{Trois premi�res it�rations de la power method}
\subsection{It�ration n�1}
\[
\begin{pmatrix} 
\frac{1051}{4550} & \frac{391}{1950} & \frac{527}{1950} & \frac{191}{1050} & \frac{794}{6825}
\end{pmatrix}
\]
\subsection{It�ration n�2}
\[
\begin{pmatrix} 
\frac{43003}{182000} & \frac{2121}{10000} & \frac{27683}{113750} & \frac{35673}{182000} & \frac{20429}{182000}
\end{pmatrix}
\]
\subsection{It�ration n�3}
\[
\begin{pmatrix} 
\frac{67623}{288557} & \frac{93767}{451224} & \frac{145393}{568750} & \frac{188765}{996486} & \frac{15789}{140000}
\end{pmatrix}
\]

\section{Score PageRank}
\[
\begin{pmatrix} 
\frac{139718}{594991} & \frac{200248}{958723} & \frac{50534}{200589} & \frac{154407}{805610} & \frac{112253}{995910}
\end{pmatrix}
\]

\appendix %annexes

\chapter{Code complet}
\lstinputlisting [language=Python, breaklines=true] {pakerank.py}

\end{document}