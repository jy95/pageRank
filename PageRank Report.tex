\documentclass[a4paper, titlepage]{report}

\usepackage[latin1]{inputenc}
\usepackage[T1]{fontenc}
\usepackage{geometry} %marges
\usepackage[francais]{babel}
\usepackage{graphicx}
\usepackage{verbatim}
\usepackage{listings} %affichage du code

\title{SINF 1250: Rapport de projet}
\author{Maxime Dillion \and Jacques Yakoub}
\date{24 d�cembre 2017}

\pagestyle{headings}

\begin{document}

\maketitle
\tableofcontents

\chapter{Th�orie}
	
\section{Rappel}
\paragraph{}
[Rappel de la partie th�orique]

\subsection{Syst�me d'�quations lin�aires}


	
\section{Calcul}
\subsection{Syst�me d'�quation sous forme matriciel}
% & \\
\[ \left( \begin{array}{cccccc}
1 & 1 & 1 & 1 & 1 & 1 \\ \\
\frac{-49}{50} & \frac{187}{350} & \frac{49}{200} & \frac{1}{50} & \frac{47}{100} & 0 \\ \\
\frac{11}{50} & \frac{-49}{50} & \frac{49}{200} & \frac{1}{5} & \frac{47}{100} & 0 \\ \\
\frac{21}{50} & \frac{1}{50} & \frac{-49}{50} & \frac{37}{50} & \frac{1}{50} & 0 \\ \\
\frac{3}{25} & \frac{97}{350} & \frac{79}{200} & \frac{-49}{50} & \frac{1}{50} & 0 \\ \\
\frac{11}{50} & \frac{26}{175} & \frac{19}{200} & \frac{1}{50} & \frac{-49}{50} & 0
 \end{array} \right)\]
\subsection{Syst�me d'�quation apr�s r�solution}
\[ \left( \begin{array}{cccccc}
1 & 0 & 0 & 0 & 0 & \frac{278338891485003}{1185309949623550} \\ \\
0 & 1 & 0 & 0 & 0 & \frac{1574961688097759}{7540409867060228} \\ \\
0 & 0 & 1 & 0 & 0 & \frac{3332146709283619}{1.32265796170871e+16} \\ \\
0 & 0 & 0 & 1 & 0 & \frac{534623}{2789366} \\ \\
0 & 0 & 0 & 0 & 1 & \frac{1572003}{13946830} \\ \\
0 & 0 & 0 & 0 & 0 & 0 \end{array} \right)\]
On a donc :
$x_{1} = \frac{278338891485003}{1185309949623550}$ ;
$x_{2} = \frac{1574961688097759}{7540409867060228}$ ;
$x_{3} = \frac{3332146709283619}{1.32265796170871e+16}$ ;
$x_{4} = \frac{534623}{2789366}$ ;
$x_{5} = \frac{1572003}{13946830}$

\newpage

\chapter{Impl�mentation}

\section{Matrice d'adjacence}
\[ \left( \begin{array}{ccccc}
0 & 2 & 4 & 1 & 2 \\
4 & 0 & 0 & 2 & 1  \\
3 & 3 & 0 & 5 & 1  \\
0 & 1 & 4 & 0 & 0  \\
3 & 3 & 0 & 0 & 0  \end{array} \right)\] 

\section{Degr� entrant des noeuds}
\[ \left( \begin{array}{ccccc}
10 & 9 & 8 & 8 & 4  \end{array} \right)\] 

\section{Matrice de probabilit� de transition}
\[ \left( \begin{array}{ccccc}
0 & \frac{4}{7} & \frac{1}{4} & 0 & \frac{1}{2} \\
\frac{2}{9} & 0 & \frac{1}{4} & \frac{1}{5} & \frac{1}{2}  \\
\frac{4}{9} & 0 & 0 & \frac{4}{5} & 0  \\
\frac{1}{9} & \frac{2}{7} & \frac{5}{12} & 0 & 0  \\
\frac{2}{9} & \frac{1}{7} & \frac{1}{12} & 0 & 0  \end{array} \right)\] 

\section{Matrice Google}
\[ \left( \begin{array}{ccccc}
\frac{1}{50} & \frac{11}{50} & \frac{21}{50} & \frac{3}{25} & \frac{11}{50} \\
\frac{187}{350} & \frac{1}{50} & \frac{1}{50} & \frac{97}{350} & \frac{26}{175}  \\
\frac{49}{200} & \frac{49}{200} & \frac{1}{50} & \frac{79}{200} & \frac{19}{200}  \\
\frac{1}{50} & \frac{1}{5} & \frac{37}{50} & \frac{1}{50} & \frac{1}{50}  \\
\frac{47}{100} & \frac{47}{100} & \frac{1}{50} & \frac{1}{50} & \frac{1}{50}  \end{array} \right)\] 

\section{Trois premi�res it�rations de la power method}
\subsection{It�ration n�1}
\[ \left( \begin{array}{ccccc}
\frac{1051}{4550} & \frac{391}{1950} & \frac{527}{1950} & \frac{191}{1050} & \frac{794}{6825}  \end{array} \right)\] 
\subsection{It�ration n�2}
\[ \left( \begin{array}{ccccc}
\frac{43003}{182000} & \frac{2121}{10000} & \frac{27683}{113750} & \frac{35673}{182000} & \frac{20429}{182000}  \end{array} \right)\]
\subsection{It�ration n�3}
\[ \left( \begin{array}{ccccc}
\frac{67623}{288557} & \frac{93767}{451224} & \frac{145393}{568750} & \frac{188765}{996486} & \frac{15789}{140000}  \end{array} \right)\] 

\section{Score PageRank}
\[ \left( \begin{array}{ccccc}
\frac{139718}{594991} & \frac{200248}{958723} & \frac{50534}{200589} & \frac{154407}{805610} & \frac{112253}{995910}  \end{array} \right)\] 

\appendix %annexes

\chapter{Code complet}
\lstinputlisting [language=Python, breaklines=true] {pakerank.py}

\end{document}